\subsection*{3.1 Processes (\code{ps}):}
\begin{enumerate}
    \item A process inside container $c$ can only see the processes in the container $c$ and any process on host cannot see any process inside a container.
    \item This isolation is implemented by \code{container\_id} of each process.
    \item \textbf{Function} \code{ps}: If processer $p$ calls \code{ps()} then it only prints the processes which has same \code{container\_id} as of $p$.
\end{enumerate}

\subsection*{3.2 File System (\code{ls}):}

\begin{enumerate}
    \item Resource Isolation in files system to implement virtual file system is obtained by associating the files corresponding to a process with its container id.
    \item Every file created by a process running in a particular container has attached with it the container id, which uniquely identifies its locality w.r.t the container. Filename is modified in the following manner for isolation:\\
     \hspace*{2cm}   \code{filename} $\rightarrow$ \code{filename\$cid} \\
     \hspace*{4cm} (\code{cid} = Container Id, cid is to be assumed with 2 digits)
    \item \textbf{ls }:
    \begin{itemize}
        \item On a process in a container : With the implementation of virtual file system as defined above, ls command only outputs the files that are local to the current container(which contains \code{\$cid} in filename), which gives the appearance of virtualization required.
        \item Careful implementation is done to prevent duplication printing in case of same file existing on the host(before joining the container) and within the container.
        \item On host : print all the files which were created on the host.
    \end{itemize}
    
\end{enumerate}

% \item  \textbf{Copy on Write (COW) }:
%         \begin{itemize}
%             \item 
%         \end{itemize}
\begin{enumerate}
    \item Kernel's scheduler is being modified to work for containers as well.
    \item Implementation of virtual scheduler can be understood as consisting of two round-robin schedulers. The main scheduler acts on all the kernel processes and containers treating them all as a single unit. 
    \item Every container also has it's own scheduler which maintains the last process that got scheduled within the container and also implements a round-robin approach within the container.
    \item This implementation ensures that all that all processes and containers get equal scheduling and processes within a container also get equal scheduling.
    \item This scheduling strategy is fair as kernel cannot see the processes inside the containers.
    \item Consider the following example with two containers and 8 processes:
    \begin{itemize}
        \item $p0,p1$ in host
        \item $p2,p3$ in container $c1$
        \item $p4,p5,p6$ in container $c2$
    \end{itemize}
    Scheduling : $p0, p1, p2, p4, p0, p1, p3, p5, p0, p1, p2, p6 ...$
\end{enumerate}

\subsection*{Log Calls:}
\lstinputlisting[language=C]{f2.c}
It is implemented by keeping a toggle variable:
\lstinputlisting[language=C]{f3.c}